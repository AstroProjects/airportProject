\chapter{Aeronautical limitation surfaces}

	\section{Physical limitation surfaces}
	\paragraph{}The aims of this section are to define the airspace around aerodromes to be	maintained free from obstacles to permit the intended aeroplane operations at the aerodromes to be done safely.
	As runaway 1 and 2 are both classified as category 4, the dimensions of the Aeronautical Servitudes are the same for both. Using the defined OBSTACLE LIMITATION SURFACES (figure \ref{servitudeMap}), on ATTACHMENT B of \cite{Standards2016}.
	
	\begin{figure}[H]
		\centering
		\includegraphics[clip, trim=0cm 0cm 0cm 0cm, width=\textwidth]{./images/servidumbres/servitudeMap}
		\caption{Obstacle Limitation Surface model.}
		\label{servitudeMap}
	\end{figure}

	Due to the fact that the table used in order to calculate and define each limitation surface will occupy a lot of space, it has been moved to the attachments. Furthermore, a little explanation and more images in order to gain a better understanding on how do limitation surfaces works has also been done. 
	
	Moving on the servitudes final value for each runway, they are gathered in the following table and represented on figure \ref{3Dservidumbres}.
	
	\begin{longtable}[htb]{@{}lr@{}}
		\toprule[3pt]
		\multicolumn{2}{c}{\textbf{\large RUNWAYS 1 \& 2  } }\\ \midrule[2pt]
		\multicolumn{2}{c}{\textbf{Conic} }\\
		\midrule[0.5pt]
		Slope & 5\%\\
		Height & 100m\\
		\midrule[2pt]
		\multicolumn{2}{c}{\textbf{Internal Horizontal} }\\
		\midrule[0.5pt]
		Height & 45m\\
		Radius & 4.000m\\
		\midrule[2pt]
		\multicolumn{2}{c}{\textbf{Internal Approximation} }\\
		\midrule[0.5pt]
		Width & 120m\\
		Distance from edge & 60m\\
		Length & 900m\\
		Slope & 2\% \\
		\midrule[2pt]
		\multicolumn{2}{c}{\textbf{Approximation} }\\
		\midrule[0.5pt]
		Inner band length & 300m\\
		Distance from edge & 60m\\
		Divergence & 15\% \\
		\midrule[0.5pt]
		\multicolumn{2}{c}{First section} \\
		\midrule[0.5pt]
		Length & 3.000m\\
		Slope & 2\%\\
		\midrule[0.5pt]
		\multicolumn{2}{c}{Second section} \\
		\midrule[0.5pt]
		Length & 3.600m\\
		Slope & 2.5\%\\
		\midrule[0.5pt]
		\multicolumn{2}{c}{Horizontal section} \\
		\midrule[0.5pt]
		Length & 8.400m\\
		Total length & 15.000m\\
		\midrule[2pt]
		\multicolumn{2}{c}{\textbf{Transition} }\\
		\midrule[0.5pt]
		Slope & 14.3\%\\
		\midrule[2pt]
		\multicolumn{2}{c}{\textbf{Inner Transition} }\\
		\midrule[0.5pt]
		Slope & 33.3\%\\
		\midrule[2pt]
		\vspace{5cm}&\vspace{5cm}\\
		\midrule[2pt]
		\multicolumn{2}{c}{\textbf{Interrupted Landing} }\\
		\midrule[0.5pt]
		Inner band length & 120m\\
		Distance from edge & 1800m\\
		Divergence & 10\%\\
		Slope & 3.3\%\\
		\midrule[2pt]
		\multicolumn{2}{c}{\textbf{Take-off Climb} }\\
		\midrule[0.5pt]
		Inner band length & 180m\\
		Distance from edge & 60m\\
		Divergence & 12.50\%\\
		Final width & 1.200m\\
		Length & 15.000m\\
		Slope & 2\% (a)\\
		\bottomrule[3pt]
		\caption{Runways 1 \& 2 servitudes. (a): It is recommended to limit the presence of new objects to a surface with a slope of 1.6\%}
	\end{longtable}
	
	The obstacle free surfaces are the same for the two runways, since they are both of category III. All the data in the table is required in order to achieve this category of precision approximation.
	
	After study of the terrain, it is concluded that there are no obstacles in inside the surfaces. The control tower is below the conic surface as well.
	
	
	\begin{figure}[H]
		\centering
		\includegraphics[clip, trim=0cm 0cm 0cm 0cm, width=0.45\textwidth]{./images/servidumbres/3Dservidumbres}
		\caption{Servitudes 3D view.}
		\label{3Dservidumbres}
	\end{figure}

	
	\section{ILS limitation surfaces}
		\section{Localizer limitation surfaces}
		\section{Gliding trajectory protection limitation surfaces}