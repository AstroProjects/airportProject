\chapter{Runway design}

	\section{Introduction}
	\paragraph{} The runway is a key piece of the airport and the air side design due to the fact that defines the maximum dimensions of the operative airplanes. Its function is to successfully guarantee the landing and takeoff operation.
	
	In order to successfully define the runway, the instructions and requirements given in the Annex 14 given by the OACI have been followed. 

	\section{Runway Length}
	\paragraph{} The reference field length is defined as the minimum length needed in order to perform a takeoff operation with the maximum homologated takeoff weight using sea level conditions, without wind and considering 0° slope.  

	In order to calculate the runway length, the first step is to choose the most restrictive airplane that is going to operate on that runway. Afterwards, that length has to be corrected by the runway slope, the altitude of the airport and the mean temperature.

	Due to the fact that the airport has two runways, from now on, the runway used for international flights will be referred as runway 1 and the one used for domestic flights will be named runway 2. 
	
		\subsection{Runway 1}
		The biggest airplanes that will operate on this runway are the Boeing 777-300ER and the Airbus A330-300. 
		
			\subsubsection{Reference Field Length}
			Starting with the B777-300ER, using the ACAP (Aircraft Characteristic for Airport Planning), the values of the MTOW (Maximum TakeOff Weight) and MLW (Maximum Landing Weight) can be obtained. 
			
	\section{Runway width}
		\subsection{Runway width for reference aircraft}
		\subsection{Final runway width}
		
	\section{Reference code}
	
	\section{Declared distances}
	
	\section{Protection and safety areas}
		\subsection{Runway shoulders}
		\subsection{Runway strips}
		\subsection{Runway end safety area (RESA)}
		\subsection{Stopway (SWY)}
		\subsection{Clearway (CWY)}
		