\chapter{Structural typology description}
When it comes to building the terminal, it is important to specify which kind of materials are going to be used, and thus, the efforts that will hold the structure. In a general frame, it can be distinguished between two types of mechanisms of effort transmission to the foundation; firstly, the vertical efforts due to the gravitational load and, secondly, the lateral efforts due to the wind or possible earthquakes.

In the first case, the load-bearing elements work in compression and the ceilings work in bending-shear in the manner of a beam subjected to perpendicular loads in its plane.

In the second case, the same elements act differently. The ceilings receive and accumulate horizontal forces, working as a membrane and distributing forces between the different vertical elements, while the vertical elements work to bending-shear.

	\section{Foundation}
Indonesia is a country with a very varied terrain due to the numerous geological faults combined with significant erosion. Volcanoes are the spine of Java. This irregular chain of volcanoes, which spread over the entire length of the island, forms the most active part of the 'Ring of Fire' volcano chain. The volcanism typical of Java's Island corresponds to the alkaline series, in which basalts, tephrites and phonolites predominate. Moreover, the zone is full of subterranean activity, earthquakes are fairly common. Most of them are not very powerful but it is very important to take into account this phenomena in terms of building the foundations.

The Island of Java is a region located in a area close to the coast, which implies the abundance of swampy land. Therefore, a deep foundation will be carried out in front of a medium or superficial one, since the superficial terrain will not be able to absorb the efforts that will transmit through a direct foundation.


	\section{Vertical elements}
	
	\section{Forge}