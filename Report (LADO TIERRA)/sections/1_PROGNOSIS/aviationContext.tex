\section{Aviation context in Indonesia}
\paragraph{} Indonesia is the fourth most populated country in the world and the largest economy in
south-east Asia.\\

Indonesia is a growing touristic destination.\\

It’s topography which is composed by many island makes essential the domestic air
transport.\\

Jakarta (located at the island of Java) is the center of government, commerce and industry of Indonesia.\\

Currently Jakarta has an international airport (Soekarno-Hatta International Airport). It operates around the 250\% over its design capacity and an interesting fact is that last year 40\% of its flights were delayed. Efforts have been made to decrease this problem opening a small military airport for civilian domestic flights. Nevertheless, the problem still persists. \\

The current Jakarta airport cannot be expanded, due to nearby neighborhoods.

"Some news have been recently published by Jakarta authorities confirming the urge of a new airport around Jakarta to absorb the saturated traffic of the Soekarno-Hatta International Airport, even after constructing a new runway and terminal on it. 
All in all, it is essential to construct a new airport. Moreover, the secondary airport, really small, is only focused on military and private services and does not have enough fields at its surroundings to expand, as Jakarta air traffic requires." [Falta cita]


	\subsection{Airport location}
\paragraph{}The main idea to find a good location was to put the new Jakarta airport in an area not too far from the city with enough space to build a big airport which has opportunities to expand in a further future.

Following this parameters, the location chosen geographically is situated to the east of the city of Jakarta at 32 km from the city center. It is also located above the emerging city of Bekasi that in the last years is increasing its industry hosting several multinationals. In addition, the terrain is not edified yet and extensive, plus it is non-mountainous and obstacles-free.

\begin{figure}[H]
	\centering
	\includegraphics[clip, trim=0cm 0cm 0cm 0cm, width=1.00\textwidth]{./images/PROGNOSIS/airport_location}
	\caption{Histograma de las medidas del diámetro inferior.}
	\label{hist_inf}
\end{figure}

It is a huge free obstacle flat field, without relevant slope gradients and the terrain is not edified yet Exceptional location due to its huge amount of terrain available where companies could settle down taking advantage of the airport proximity, low terrain costs and direct connection with the down town. Enough space to become also a logistic distribution center of the island and Indonesia.\\

Finally, as it is an almost virgin land, communication is limited. Therefore, the solution is easy. The present Jakarta motorway will be extended. As it is shown on the screen, there will be two connections between the current and the new highway. \\

Connection between airports will be done through the recently built highway. It will take 40min from door to door. Connection network of free-busses between both airports will make transfers safe and easy. There will be also available busses to and from the city centre, at low prices. During rush hours, a specific way will be delimited only for airport bus transfers. 


	
	\subsection{Current traffic}
	\subsection{Occupation factor}