\section{Forecast computation}
According to the project objective, it is necessary a first step of performing a traffic forecast study for the airport which is being projected. This step is the most important one since the terminal building dimensions will directly depend on the passenger demand and the platform area will directly depend on the aircraft or operations demand.

Before starting the prediction of the demands mentioned above, some important aspects will be taken into account:

The airport to be projected will absorb nearly the 40(percent) of Soekarno-Hatta’s current traffic, which corresponds to the percentage of delayed flights due to congestion.

The growth tendencies, concretely the Compound Annual Growth Rate (CAGR), has been calculated using the data corresponding to the five more recent years.

The predictions of passengers and operations have been calculated using three possible scenario.

Neutral scenario: The growth is constant along time

Optimistic scenario: The growth is constant along time but in 2025 is multiplied by a factor of 1,5 due to, for example, a boom of tourism in the city.

Pessimistic scenario: The growth is constant along time but in 2025 is divided by a factor of 2 due to, for example, a period of time with adverse atmospheric phenomena.



	\subsection{Passenger prediction}
	\subsection{Operations prediction}
	\subsection{Flights on standard day}