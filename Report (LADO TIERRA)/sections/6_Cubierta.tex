\chapter{Building cover}

The building cover is the constructive element which protects it at the top and, by extension, the supporting structure of the element itself. Before choosing the best solution for the terminal building of Bekasi-East Jakarta Airport, it is important to remember the features expected from a building cover:

\begin{itemize}
	\item Resistance and stability, the cover has to support its own weight and possible overloads due to snow, wind, etc.
	\item Atmospheric barrier outdoors, must protect from water, wind, sun.
	\item Hydrothermal insulation, a vapour barrier must be installed to prevent condensation inside the thermal insulation.
	\item Acoustic barrier, must be protected from aircraft noise and vibrations.
	\item Thermal and fire insulation.
\end{itemize}

In addition, the cover also has to be aesthetic when facing the users of airports. In other areas you never see the building covers, but in this case many customers will arrive by plane and the cover will the first thing to be seen when arriving at the airport.

	\section{Solution adopted}
In this terminal building, a metal cover with sandwich structure is chosen as the best solution. It has many advantages in front other types of covers involving the covering, supporting structure, insulation and aestheticism of both inside and outside the building. The core made of expanded polyurethane allows to obtain a fire resistance of 2 hour. In addition, these covers are able to support great bending efforts thanks to the design of the panels and its thickness. As soon as the installation is concerned, also thanks to the composite structure it has, it allows to install the cover, insulating elements and other final details in a single stage. In this way, the reduction of the costs is significant.
	
	\section{Shape and inclination of the building cover}
	
The solution adopted for the shape and inclination of the cover, taking into account the rainy climate in the region of Indonesia, is to build a mixed cover. In the centre part and concerning all the length of the pier, the cover will be curved while in the rest of the terminal an inclined cover of 5 degrees will be used.

This type of covers are the most efficient ones when evacuating water because they do not generate leak-tightness and no extra support is necessary since gravity acts on its own. Moreover, the maintenance is not so much and consequently the costs generated due to this are lower than in other kind of covers. The only disadvantage it has the impossibility of installing machinery above the cover.

	\section{Used materials}
		
	\section{Sewer system}